\documentclass[elsart12,eqsecnum,graphics,cite,nofootinbib]
%{revtex4-2}
{revtex4-2}


 %
%
%%%%%%%%%%%%%%%%%%%%%%%%%%%%%%%%%%%%%%%%%%%%%%%%%%%%%%%%%%%%%%%%%%%%%%%%%
%\documentstyle[elsart12,osa,epsf,eqsecnum,showkeys]{revtex4}
%\documentclass[elsart12,epsf,eqsecnum]{revtex4}
%\documentclass[elsart12,epsf,eqsecnum]{article}
%\documentstyle[elsart12,osa,epsf,eqsecnum,epsfig,graphics,showkeys,cite]{revtex}
%\usepackage{showkeys}

\usepackage{epsfig,epsf}
\usepackage{graphicx}
\usepackage{grffile}
\usepackage{amsmath,amsfonts,amssymb,amsthm,nccmath,latexsym,mathtools}
\usepackage[mathscr]{euscript}
\usepackage{color}
\usepackage{hyperref}
\DeclareMathOperator{\Tr}{Tr}
\usepackage{lineno}
\usepackage{slashed}
\usepackage{array}
\usepackage{graphicx}% use this package if an eps figure is included.
\usepackage{mathrsfs}
\usepackage{multirow}
\usepackage{siunitx}
\usepackage{float}
\usepackage{hyperref}
\hypersetup{
    colorlinks,
    citecolor=blue,
    filecolor=blue,
    linkcolor=blue,
    urlcolor=blue
}
\usepackage{bm}
\usepackage{academicons}

% Feynman Diagram
\usepackage{pgfplots}
\usepackage{tikz}
\usepackage{amsmath}
\usepackage{feynmp-auto}
\usepackage[compat=1.1.0]{tikz-feynman}
\usetikzlibrary{decorations.pathmorphing}


\renewcommand{\v}[1]{ \ensuremath{ {\bm{#1}} }} 
\newcounter{questioncounter}
\newcounter{equestioncounter}
\setlength\parskip{10pt} \setlength\parindent{0in}
\newcommand{\no}{\noindent}
\begin{document}


%\hoffset-1cm
%\today
% Yale printer values
%\voffset1.5cm

%\draft{BI-TP 2005/19}
%\preprint{BI-TP 2005/19, CERN-PH-TH-2005/99
\title{Path integral formulation of diffusion model}

\author{Haowu Duan$^1$}
\affiliation{
$^1$ Physics Department, University of Connecticut, 2152 Hillside Road, Storrs, CT 06269, USA\\
}
%\date{\today}



%\begin{abstract}\end{abstract}


\maketitle

\tableofcontents

%%%%%%%%%%%%%%%%%%%%%%%%%%%%%%%%%%%%%%%%%%%%%%%%%%%%%%%%%%
\newpage
\section{Path integral formulation of the forward and time-reversed process}
    Flow model and diffusion model can be understood as the time evolution of the initial noise toward the target data distribution either through
ordinary different equation or stochastic differential equation. In addition, stochastic process can be formulated with path integral. In this note,
we will formulate the flow model and diffusion model using path integral to gain insight into the machine learning process. 

We start with the forward process with a slightly different notation.
\begin{equation}
    \begin{split}
        dx(t)=f(t, x(t))\; dt + g(t)\; dW_t
    \end{split}
\end{equation}    
where $f(t, x(t))$ is the drift term and $g(t)$ is the scale of the noise. Let's denote the noise term as $\xi_t$











A discretized version
is,
\begin{equation}
    \begin{split}
        \Delta x(t)=f(t, x(t))\; \Delta t + g(t)\sqrt{\Delta t} v_t
    \end{split}
\end{equation} 
where $v_t$ is drawn from $N(0, 1)$. We assume there is no correlation between noises from different time steps.
\begin{equation}
\begin{split}
\langle v_t v_{t'} \rangle =\delta_{t,t'}
\end{split}    
\end{equation}    
a different perspective of this process is that, probability distribution at time $t+\Delta t$ is given by,
\begin{equation}
\begin{split}
&P(t+\Delta t, x(t+\Delta t))\\
=&\frac{1}{\sqrt{2\pi}}\int d v_t e^{-\frac{v_t^2 \Delta_t}{2g(t)^2}}\int dx(t)\; P(t, x(t))\; \delta\Big(x(t)-x_{sol}(t)\Big)\\
\end{split} 
\end{equation}    
where we have assumed that we know $P(t,x(t))$, we also have $x_{sol}(t)$ to be the solution of the equation of motion.
\begin{equation}
\begin{split}
    \delta\Big( x(t+\Delta t)- x(t)-f(t,x(t))\Delta t + g(t) \sqrt{\Delta t} v_t\Big)
\end{split}
\end{equation}    


\end{document}